\begin{align} \label{2/2/3eq1}
\Vec{Q-P} & =\myvec{-9 \\-3 \\} -\myvec{5 \\2}\\
 & = \myvec{-14 \\ -5}
 \label{2/2/3eq2}
\Vec{R-P} & =\myvec{-3 \\-5 } -\myvec{5 \\2}\\
 & = \myvec{-8 \\ -7}
\end{align}


%     As the vector cross product of two vectors can also be expressed as the product of a skew-symmetric matrix and a vector.

% \begin{align} \label{2/2/3eq4}
%     \Vec{A}\times  \Vec{B}  = 
%     \myvec{
%     0&-a_{3}&a_{2}\\
%     a_{3}&0&-a_{1}\\
%     -a_{2}&a_{1}&0\\
%     }
%     \times \myvec{
%     b_{1}\\
%     b_{2}\\
%     b_{3}\\}
% \end{align}

%     Substituting values from equation \ref{2/2/3eq1} and \ref{2/2/3eq2} in above equation \ref{2/2/3eq4}, we'll get:
\begin{align} 
%\label{2/2/3eq5}
\because (\Vec{Q-P})\times ( \Vec{R-P} ) & = \myvec{0&0&-5\\0&0&14\\5&-14&0\\} \times \myvec{-8\\-7\\0\\}\\
 & = \myvec{0\\0\\58\\}
%\label{2/2/3eq5}
\implies \norm{(\Vec{Q-P})\times ( \Vec{R-P} )} =\sqrt{0^2 + 0^2+ 58^2} = 58,
\end{align}
area of the triangle is given by 
\begin{align} 
%\label{2/2/3eq3}
 \frac{1}{2}\norm{ (\Vec{Q-P})\times ( \Vec{R-P} )} = \frac{1}{2}(58) =29 
 \end{align}


% Substituting value from equation \ref{2/2/3eq5} in equation \ref{2/2/3eq3}, we'll get area of triangle:

% $\implies  units^2$

